\documentclass{article}
\usepackage{amsmath}
\usepackage{listings}

\author{Artem Shinkarov}
\date{\today}
\title{Closed formula for $a_{n} = \beta a_{n-1} + \gamma$ expression}

\begin{document}
\maketitle

\section{Introduction}
In this documet we are going to find the closed formula for the recuurent
expression of the following forms.
\par\noindent
\begin{tabular}{p{.48\textwidth}p{.48\textwidth}}
\begin{eqnarray*}
  a_0 &=& \alpha \\
  a_n &=& \beta a_{n-1} + \gamma, \forall n \geq 1
\end{eqnarray*}
&
\begin{eqnarray*}
  a_0 &=& \alpha \\
  a_n &=& \dfrac{1}{\beta} a_{n-1} + \gamma, \forall n \geq 1
\end{eqnarray*}
\end{tabular}

\par\noindent
We are going to build a closed representation using generating functions.

\section{First case}
First, we will consider the case with multiplication, find a closed formula
and use the different coefficient. We are going to use ordinary generating
function
\begin{equation}
  G(z) = \sum\limits_{n = 0}^{\infty} {a_n z^n} \label{eq:genfunc}
\end{equation}

We start with multiplying every equation with $z^n$ and then adding them
together.

\begin{eqnarray*}
  z^0 a_0 &=& z^0 \alpha \\
  z^n a_n &=& z^n \left(\beta a_{n-1} + \gamma\right), \forall n \geq 1 \\
  G(z) &=& \alpha + \beta\sum\limits_{n = 1}^{\infty}{a_{n-1}z^n} + 
	   \gamma\sum\limits_{n=1}^{\infty}z^n
\end{eqnarray*}

Now we are going to use the following facts.
\begin{equation}
  \sum\limits_{n=1}^{\infty}{a_{n-1}z^n} 
    = z\sum\limits_{n=1}^{\infty}{a_{n-1}z^{n-1}}
    = z\sum\limits_{n=0}^{\infty}{a_n z^n} 
    \label{eq:fact1}
\end{equation}

and 

\begin{equation}
  \sum\limits_{n=0}^{\infty}{z^n} = \dfrac{1}{1 - z} 
  = 1 + z + z^2 + \cdots
  \label{eq:fact2}
\end{equation}

Using facts (\ref{eq:fact1}) and (\ref{eq:fact2}) we would rewrite the 
formulas in the following way.
\begin{eqnarray*}
  G(z) &=& \alpha + \beta z G(z) + 
	   \gamma \left(\dfrac{1}{1-z} - 1\right) \\
  G(z) &=& \beta z G(z) + \dfrac{\gamma}{1-z} + \alpha - \gamma \\
  G(z)(1 - \beta z) &=& \dfrac{\gamma}{1-z} + \alpha - \gamma \\
  G(z) &=& \dfrac{\gamma}{(1-z)(1 -\beta z)} + \dfrac{\alpha}{1 - \beta z}
	   - \dfrac{\gamma}{1 - \beta z} \\
  G(z) &=& \dfrac{\gamma}{1 - \beta} 
	   \left(\dfrac{1}{1-z} + \dfrac{-\beta}{1 - \beta z}\right)
	   + \left(\alpha - \gamma\right) \dfrac{1}{1 - \beta z}
\end{eqnarray*}

Now we will use the following fact to find the closed representation.
\begin{equation}
  \dfrac{1}{1 - \alpha z} = \sum\limits_{n = 0}^{\infty}{\alpha^n z^n}
  \label{eq:fact3}
\end{equation}

Applying (\ref{eq:fact3}) to the $G(z)$ formula we get the following.
$$
  G(z) = \sum\limits_{n=0}^{\infty}{a_n z^n} =
	 \dfrac{\gamma}{1 - \beta} 
	 \left(\sum\limits_{n=0}^{\infty}{z^n}
	       - \beta\sum\limits_{n=0}^{\infty}{\beta^n z^n} \right)
	 + \left(\alpha - \gamma\right) 
	   \sum\limits_{n=0}^{\infty}{\beta^n z^n}
$$

Now the closed formula for $a_n$ is the following.

$$
  a_n = \dfrac{\gamma}{1 - \beta}
	   \left(1^n -\beta\beta^n\right)
	   + \left(\alpha - \gamma\right) \beta^n
$$

After some transformations:

\begin{equation}
  a_n = \dfrac{\gamma}{1 - \beta} 
	+ \dfrac{\alpha - \gamma - \alpha\beta}{1 - \beta} \beta^n
  \label{eq:mult}
\end{equation}

\section{Second case}
The second case can be expressed in terms of the first case by 
replacing $\beta$ with $\dfrac{1}{\beta_1}$. We are going to make
this substitution with (\ref{eq:mult}).

$$
  a_n = \dfrac{\gamma}{1 - \frac{1}{\beta_1}} 
	+ \left(\dfrac{\alpha - \gamma - \frac{\alpha}{\beta_1}}
		{1 - \frac{1}{\beta_1}}\right) \frac{1}{\beta_1^n}
$$

After trivial transformations we get the following closed formula
for the second case.

\begin{equation}
  a_n = \dfrac{\gamma\beta_1}{\beta_1 - 1}
	+ \dfrac{\gamma\beta_1 - \gamma\beta_1 - \alpha}{\beta_1 - 1}
	  \dfrac{1}{\beta_1^n}
  \label{eq:div}
\end{equation}

Original expression can be matched by replacing $\beta_1$ with 
$\beta$.

\section{Test the formulas}
Here is a python script to test the formulas. It mainly can be used to
avoid copy-pasting errors.

\lstinputlisting[language=python]{closed_formulas.py}
\end{document}
